\documentclass[12pt]{llncs}

\usepackage[utf8]{inputenc}
\usepackage[spanish,activeacute]{babel}
\usepackage{graphicx}
\usepackage{subcaption}
\usepackage[colorlinks, citecolor=black, urlcolor=black, bookmarks=false, hypertexnames=true]{hyperref} 
\usepackage{url}
\usepackage{listings}
\usepackage{color}

\title{Gologle}
\author{
    Samuel David Suárez Rodríguez \and
    Gabriel Fernando Martín Fernández \and
    Enmanuel Verdesia Suárez
}
\date{}
\institute{Universidad de La Habana}

\begin{document}

\maketitle

Se pretende desarrollar un Sistema de Recuperación de Información (SRI) con el apoyo de técnicas de Inteligencia Artificial (IA). Dado un conjunto de varios documentos de diversos temas se desarrollará una aplicación de búsqueda que obtenga resultados relevantes y acordes a una consulta presentada por el usuario. Además se emplearán métodos para clasificar los documentos y extraer sus características más relevantes empleando algoritmos para la clasificación y selección de características. También se desarrollarán herramientas para generar clústeres de documentos y agrupar los que tengan contenidos afines en un mismo grupo. Para este último proceso también es posible generar etiquetas que describan los clústeres generados empleando algoritmos de cluster labeling. Todos estos procesos de clasificación y agrupamiento se apoyarán en el uso de algoritmos de machine learning (ML) para lograr su objetivo.

% \bibliography{bib} 
% \bibliographystyle{ieeetr}

\end{document}